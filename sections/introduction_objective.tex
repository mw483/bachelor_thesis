\section{Introduction}

\subsection{Background}
Air pollution in urban areas represents an ongoing global challenge that negatively impacts individual health, ecosystem integrity, and places increasing pressure on medical systems and economies worldwide. Urban air pollution comprises multiple components, including carbon dioxide (CO$_2$), carbon monoxide (CO), and particulate matter with an aerodynamic diameter of less than 2.5 $\mu$m (PM$_{2.5}$). 

The health implications are severe; recent data from the \textcite{WHO2024Ambient} indicates that ambient air pollution contributes to approximately 4.2 million premature deaths globally each year. While mortality is largely attributed to increased private vehicle demand and population growth, a significant contributing factor is the lack of consideration for pollutant dispersion during rapid urban development \parencite{Li2021Review}. As noted by \textcite{Li2021Review}, "hasty and irrational construction strategies" often lead to poor ventilation, trapping pollutants in high concentrations near the ground level.

Understanding and mitigating these concentrations requires a dual-level approach. First, it involves the utilization of mechanical factors, which include forced convection by wind, natural convection by solar radiation, and traffic-induced turbulence. Second, it requires the strategic design of urban morphology. This focus on the "sound design of the urban landscape" emphasizes building geometries, urban density, and enclosure degrees to improve the city's natural capacity to flush out pollutants \parencite{Li2021Review}.

The morphology and organization of urban areas directly influence urban traffic patterns and pollutant flow dynamics. Therefore, understanding the relationship between urban morphology, pollutant transport, and air quality is critical for developing effective mitigation strategies that move beyond qualitative observations toward quantitative air quality improvements.


\newpage
\subsection{Objective}


\newpage

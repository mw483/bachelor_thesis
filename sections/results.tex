\subsubsection{Data Processing and Validation Strategy}

\begin{table}[ht]
\centering
\caption{Previous zemi, no approach. }
\label{tab:validation_no_approach}
\begin{tabular}{lrrrrrrr}
\hline
Metric & N & Hit Rate ($q$) & FAC2 & MAE & RMSE & (D) & (W) \\
\hline
$u_m/u_H$ & 90 & 0.578 & 0.756 & 0.1713 & 0.2431 & 0.25 & 0.050 \\
$v_m/u_H$ & 90 & 0.967 & 1.000 & 0.0178 & 0.0232 & 0.25 & 0.050 \\
$w_m/u_H$ & 90 & 0.289 & 0.311 & 0.0980 & 0.1165 & 0.25 & 0.050 \\
$u_{\text{rms}}/u_H$ & 90 & 0.511 & 0.644 & 0.0783 & 0.0995 & 0.50 & 0.005 \\
$v_{\text{rms}}/u_H$ & 90 & 0.544 & 0.644 & 0.0572 & 0.0680 & 0.50 & 0.005 \\
$w_{\text{rms}}/u_H$ & 90 & 0.322 & 0.678 & 0.0581 & 0.0650 & 0.50 & 0.005 \\
$\text{TKE}/u_H^2$ & 90 & 0.222 & 0.600 & 0.0281 & 0.0321 & 0.50 & 0.005 \\
\hline
\end{tabular}
\end{table}

\begin{table}[ht]
\centering
\caption{2 km approach. Calculated U$_H$ for normalization: 3.4744 m/s.}
\label{tab:validation_2km_approach}
\begin{tabular}{lrrrrrrr}
\hline
Metric & N & Hit Rate ($q$) & FAC2 & MAE & RMSE & (D) & (W) \\
\hline
$u_m/u_H$ & 90 & 0.856 & 0.944 & 0.0706 & 0.1392 & 0.25 & 0.050 \\
$v_m/u_H$ & 90 & 0.933 & 1.000 & 0.0215 & 0.0264 & 0.25 & 0.050 \\
$w_m/u_H$ & 90 & 0.489 & 0.578 & 0.0565 & 0.0686 & 0.25 & 0.050 \\
$u_{\text{rms}}/u_H$ & 90 & 1.000 & 1.000 & 0.0313 & 0.0396 & 0.50 & 0.005 \\
$v_{\text{rms}}/u_H$ & 90 & 1.000 & 1.000 & 0.0426 & 0.0519 & 0.50 & 0.005 \\
$w_{\text{rms}}/u_H$ & 90 & 0.922 & 1.000 & 0.0375 & 0.0470 & 0.50 & 0.005 \\
$\text{TKE}/u_H^2$ & 90 & 0.689 & 0.978 & 0.0255 & 0.0321 & 0.50 & 0.005 \\
\hline
\end{tabular}
\end{table}

\begin{table}[ht]
\centering
\caption{3 km approach. Calculated U$_H$ for normalization: 4.0868 m/s.}
\label{tab:my_data_3km}
\begin{tabular}{lrrrrrrr}
\hline
Metric & N & Hit Rate ($q$) & FAC2 & MAE & RMSE & (D) & (W) \\
\hline
$u_m/u_H$ & 90 & 0.722 & 0.878 & 0.0871 & 0.1563 & 0.25 & 0.050 \\
$v_m/u_H$ & 90 & 0.956 & 0.978 & 0.0221 & 0.0278 & 0.25 & 0.050 \\
$w_m/u_H$ & 90 & 0.544 & 0.544 & 0.0609 & 0.0737 & 0.25 & 0.050 \\
$u_{\text{rms}}/u_H$ & 90 & 1.000 & 1.000 & 0.0424 & 0.0525 & 0.50 & 0.005 \\
$v_{\text{rms}}/u_H$ & 90 & 1.000 & 1.000 & 0.0363 & 0.0420 & 0.50 & 0.005 \\
$w_{\text{rms}}/u_H$ & 90 & 0.956 & 1.000 & 0.0370 & 0.0419 & 0.50 & 0.005 \\
$\text{TKE}/u_H^2$ & 90 & 0.667 & 0.744 & 0.0213 & 0.0248 & 0.50 & 0.005 \\
\hline
\end{tabular}
\end{table}

\subsubsection{Validation Metric Equations}
\label{sec:equations}

Here are the formal definitions for the validation metrics, where $P_i$ is the predicted (LBM) value, $O_i$ is the observed (AIJ) value, and $N$ is the total number of data points.

\textbf{Hit Rate ($q$)}
The fraction of data points where the relative error is within a tolerance $D$ OR the absolute error is within a tolerance $W$.

\begin{equation}
q = \frac{1}{N} \sum_{i=1}^{N} n_i
\end{equation}
where
\begin{equation}
n_i = \begin{cases}
1 & \text{if } \left| \frac{P_i - O_i}{O_i} \right| \le D \quad \text{or} \quad |P_i - O_i| \le W \\
0 & \text{else}
\end{cases}
\end{equation}

\subsection{Factor of Two (FAC2)}
The fraction of data points where the ratio of prediction to observation is between 0.5 and 2.0, OR *both* the prediction and observation are below a low-value threshold $W$.

\begin{equation}
\text{FAC2} = \frac{1}{N} \sum_{i=1}^{N} n_i
\end{equation}
where
\begin{equation}
n_i = \begin{cases}
1 & \text{if } 0.5 \le \frac{P_i}{O_i} \le 2.0 \quad \text{or} \quad \left( |P_i| \le W \text{ and } |O_i| \le W \right) \\
0 & \text{else}
\end{cases}
\end{equation}

\subsection{Mean Absolute Error (MAE)}
The average of the absolute differences between prediction and observation.

\begin{equation}
\text{MAE} = \frac{1}{N} \sum_{i=1}^{N} |P_i - O_i|
\end{equation}

\subsection{Root Mean Square Error (RMSE)}
The square root of the average of the squared differences. This metric penalizes large errors more heavily.

\begin{equation}
\text{RMSE} = \sqrt{\frac{1}{N} \sum_{i=1}^{N} (P_i - O_i)^2}
\end{equation}

\subsection{Reynolds Number}

\begin{equation}
    Re = \frac{U \cdot L}{\nu}
\end{equation}

\begin{equation*}
    Re_{exp} = 4.3\cdot 10^{4}
\end{equation*}

\begin{align*}
    Re_{LBM} = \frac{4.940\cdot64}{1.512\cdot10^{-5}} = 2.091\cdot10^7
\end{align*}

Adapted to the experiment and simulation scenario, choosing building height $H$ as the

\begin{equation}
    Re = \frac{U_H \cdot H}{\nu}
\end{equation}

\subsection{TKE (k)}

\begin{align}
u_{rms}^2 &= \overline{u^2} - \bar{u}^2 \label{eq:urms} \\
v_{rms}^2 &= \overline{v^2} - \bar{v}^2 \label{eq:vrms} \\
w_{rms}^2 &= \overline{w^2} - \bar{w}^2 \label{eq:wrms}
\end{align}

% --- Turbulent Kinetic Energy (TKE) ---
% The 'equation' environment is for a single, numbered equation.
% We use \text{TKE} to ensure "TKE" is rendered as upright text,
% not as variables T*K*E.

\begin{equation}
\text{k} = \frac{1}{2} (u_{rms}^2 + v_{rms}^2 + w_{rms}^2) \label{eq:tke}
\end{equation}

\subsection{Concentration}

\begin{equation}
C_{norm} = \frac{C_o \cdot U_H \cdot H^2}{Q}
\end{equation}

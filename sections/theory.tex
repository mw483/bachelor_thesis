\section{Theory}

\subsection{Lattice Boltzmann Method}

To simulate urban air pollutant transport, this study employs the Lattice Boltzmann Method (LBM), a mesoscopic computational fluid dynamics (CFD) technique. Unlike traditional solvers that directly discretize the macroscopic Navier-Stokes equations, LBM models fluid flow by simulating the collective behavior of a distribution of particles on a discrete grid, or "lattice" \parencite{Kruger2017LBM}.The movement of these particles is governed by the Lattice Boltzmann Equation (LBE), which describes the evolution of the discrete-velocity distribution function $f_i$:

\begin{equation}
    f_{i}(\mathbf{x} + \mathbf{c}_{i}\Delta t, t + \Delta t) - f_{i}(\mathbf{x}, t) = \Omega_i(f) + S_i    
\end{equation}

Where $f_i$ is the distribution function representing the density of particles moving with velocity components $\mathbf{c}_i$ in the $i$-th direction. The term $\Omega_i$ represents the collision operator, which models the internal redistribution of particle momentum, while $S_i$ is the source term. The inclusion of $S_i$ is critical for this research as it allows for the integration of external body forces and the modeling of mass sources, such as vehicle emissions.The simulation proceeds through a sequence of two primary operations: streaming and collision. During the streaming step, particles move from their current node $\mathbf{x}$ to adjacent points based on their respective velocities. Following this, the collision step redistributes the particle velocities at each node based on local interactions, effectively mimicking fluid viscosity and relaxation toward equilibrium. As noted by \textcite{Kruger2017LBM}, LBM offers several distinct advantages for simulating complex urban environments. First, the underlying Boltzmann equation is essentially a hyperbolic advection equation where the source terms depend only on local values of $f$. Because these operations rely strictly on information from a node and its immediate neighbors, the algorithm is inherently local. This locality facilitates high-performance parallelization, which is essential for large-scale urban simulations. Furthermore, the grid-based nature of LBM allows for the efficient handling of complex building geometries and irregular urban morphologies without the need for the intensive re-meshing often required by finite-element methods.

\subsubsection{Lattice Configuration and Accuracy}
The choice of the D3Q27 lattice model is fundamental to the accuracy of the turbulence modeling in this research \parencite{Yokouchi2020Simulation}. While the simpler D3Q19 model is often used to reduce computational costs, it lacks the rotational invariance required to capture complex secondary flows accurately. As demonstrated by \textcite{Suga2015D3Q27}, the D3Q27 lattice provides a superior representation of turbulent structures. Their comparative analysis of turbulent pipe flow showed that the D3Q27 model produces significantly more realistic flow fields than the D3Q19 model, which tends to struggle with resolving fine-scale turbulent fluctuations in three-dimensional space. By utilizing 27 discrete velocities, the model can more effectively resolve the secondary circulations and high-frequency fluctuations necessary for an in-depth understanding of the pollutant dispersion mechanism in urban areas.The internal redistribution of particle momentum is governed by the Bhatnagar-Gross-Krook (BGK) collision operator. This operator assumes that the particle populations relax toward a local equilibrium state, $f_i^{eq}$, at a rate controlled by a single relaxation time $\tau$ \parencite{Kruger2017LBM}:

\begin{equation}
    \Omega_i(f) = -\frac{f_i - f_i^{eq}}{\tau}\Delta t    
\end{equation}

The equilibrium distribution $f_i^{eq}$ is formulated as a second-order expansion of the Maxwell-Boltzmann distribution in terms of the macroscopic density $\rho$ and velocity \textbf{u}:
\begin{equation}
    f_i^{eq}(\mathbf{x},t) = w_i \rho \left( 1 + \frac{\mathbf{u} \cdot \mathbf{c}_i}{c_s^2} + \frac{(\mathbf{u} \cdot \mathbf{c}_i)^2}{2c_s^4} - \frac{\mathbf{u} \cdot \mathbf{u}}{2c_s^2} \right)    
\end{equation}

Here, $c_s$ represents the isothermal speed of sound, defined as $c_s^2 = \frac{1}{3}\frac{\Delta x^2}{\Delta t^2}$. This formulation ensures the local conservation of mass and momentum while establishing the relationship between pressure and density as $p = c_s^2 \rho$.

\subsubsection{Large Eddy Simulation and Turbulence Modeling}
To capture the transient, multi-scale nature of urban wind flows, this research utilizes a Large Eddy Simulation (LES) approach. The unresolved subgrid-scale (SGS) motions are accounted for using the Coherent-structure Smagorinsky Model (CSM). In a standard Smagorinsky framework, the turbulent viscosity $\nu_t$ is determined by the strain-rate tensor $|S|$ and a filter width $\Delta$:

\begin{equation}
    \nu_t = (C_S\Delta)^2|S|    
\end{equation}

Within the LBM framework, the total viscosity $\nu$ (the sum of molecular and turbulent viscosity) is directly linked to the relaxation time:

\begin{equation}
    \nu = c_s^2 \left(\tau - \frac{\Delta t}{2} \right)    
\end{equation}

While a standard Smagorinsky constant $C_S$ typically ranges between 0.1 and 0.2 \parencite{Blazek2015CFD}, the CSM used here employs a dynamic parameter based on the second invariant of the velocity gradient tensor. This approach allows the model to automatically account for wall-damping effects, ensuring more accurate flow predictions near building surfaces without requiring manual damping functions \parencite{Phuc2018CSM}.

\subsubsection{Implementation of Lagrangian Stochastic Model}
To simulate the transport and dispersion of particulate matter within the urban environment, this study utilizes a Lagrangian Stochastic Model (LSM). Unlike Eulerian methods that treat pollutants as a continuous concentration field, the LSM tracks the trajectories of thousands of discrete "passive" particles within the turbulent flow field generated by the LBM. This approach is particularly effective for modeling dispersion in the inhomogeneous and non-Gaussian turbulence characteristic of the urban planetary boundary layer (PBL). By simulating a statistically significant ensemble of particle paths, the ensemble-mean concentration can be derived, which is directly proportional to the particle number density or the probability density function (PDF) of particle positions \parencite{Weil2024LSM}.The trajectory of each particle is determined by integrating its position $x_p$ over time. For a particle starting at an initial position $x_0$, the position at the next time step $t + \Delta t$ is calculated as:

\begin{equation}
    x_p(x_0, t + \Delta t) = x_p(x_0, t) + u_p(x_0, t)\Delta t    
\end{equation}

where $u_p$ represents the instantaneous Lagrangian particle velocity. This velocity is composed of two primary components: $$u_p = u_r + u_s$$ 
In this formulation, $u_r$ is the resolved grid-scale (GS) velocity obtained directly from the LBM-LES results, representing the large-scale atmospheric motions. The term $u_s$ represents the subgrid-scale (SGS) velocity fluctuations, which account for the smaller, unresolved turbulent motions.The resolved velocity $u_r$ at the particle's location is determined by interpolating the LBM grid results using the weighting factors $w_i$ of the neighboring nodes: $$u_r = \sum_{i=0}^7(w_i U_i)$$

To account for the unresolved fluctuations, the SGS velocity $u_s$ is modeled as a stochastic process. Following the framework established by \textcite{Weil2024LSM} and \textcite{Thomson1987Stochastic}, the evolution of the SGS velocity $du_s$ is described by a stochastic differential equation: 

\begin{equation}
    du_s = -\frac{3f_sC_0\epsilon}{4}dt + \frac{1}{2}\left( \frac{1}{e_s}\frac{de_s}{dt}u_s + \frac{2}{3}\nabla e_s \right)dt + (f_sC_0\epsilon)^{1/2}d\xi
\end{equation}

where $e_s$ is the local SGS turbulent kinetic energy (TKE) and $f_s$ represents the mean contribution of the SGS TKE to the total TKE. The term $\epsilon$ denotes the total dissipation rate, while $C_0$ is a universal constant (set to 4.0 in this study). The stochastic component $(f_sC_0\epsilon)^{1/2}d\xi$ incorporates Gaussian white noise via the displacement vector $d\xi$, allowing the model to represent the random nature of sub-lattice turbulent dispersion.

\subsubsection{Advanced Wall Boundary Conditions}
In LBM, the standard Bounce-Back (BB) scheme assumes that fluid particles hitting a solid boundary reverse their momentum, effectively enforcing a no-slip condition ($\mathbf{u}=0$) at the wall. While numerically stable and simple to implement, the BB scheme requires the first grid point to reside within the viscous sublayer to accurately capture wall shear stress. In urban-scale simulations, the grid resolution is typically much larger than the thickness of this sublayer, causing the BB scheme to underpredict shear drag.To resolve this, a Wall Function Boundary (WFB) is implemented, following the approach evaluated by \textcite{Han2021WFB}. The WFB utilizes Spalding's Law to model the non-linear velocity profile in the logarithmic layer, allowing the simulation to approximate near-wall velocity without fully resolving the thin viscous region. This modification is applied during the collision step by introducing a momentum loss term based on the wall shear stress $\tau_w$:

\begin{equation}
    f_a^* = f_a \pm \frac{\Delta t}{2\Delta y}\tau_{w,j}
\end{equation}

where $\tau_{w,j}$ is the shear stress in the $j$-direction ($x$ or $z$). The wall shear stress is determined by solving Spalding's implicit equation:

\begin{equation}
    y^+ = u^+ + e^{-\kappa B} \left[ e^{\kappa u^+} - 1 - (\kappa u^+) - \frac{(\kappa u^+)^2}{2} - \frac{(\kappa u^+)^3}{6} \right]
\end{equation}

Here, $y^+$ is the dimensionless distance from the wall, and $u^+$ is the dimensionless velocity. By calculating the friction velocity $u_{\tau} = \sqrt{\tau_w / \rho}$ from this relationship, the model applies a "reverse resistance" that decelerates particles near building surfaces. As demonstrated by \textcite{Han2021WFB}, this WFB approach significantly improves the accuracy of shear drag predictions in high-Reynolds-number urban flows where coarse grids are a necessity.

\subsubsection{Subgrid-Scale Turbulent Kinetic Energy (SGS TKE)}
The calculation of the SGS TKE ($e_s$) is vital for the Lagrangian Stochastic Model, as it dictates the magnitude of the stochastic velocity fluctuations. While traditional LES often parameterizes $e_s$ via the eddy viscosity and a mixing length ($v_t = c_k l e_s^{1/2}$), this study utilizes a filtering approach as implemented by \textcite{Suga2015D3Q27} and \textcite{Yokouchi2020Simulation}. The SGS TKE ($k_{sgs}$) is derived from the difference between the instantaneous grid-scale velocity and a filtered velocity field:

\begin{equation}
    k_{sgs} = C_{kes} \sum_{i=1}^3 (\langle \hat{U}_i \rangle - \langle U_i \rangle)^2    
\end{equation}


The filtered velocity $\langle \hat{U}_i \rangle$ is calculated using a weighted average of the central node (50\% weight) and its six immediate neighbors (1/12 weight each):

\begin{equation}
    \langle \hat{U}_i \rangle = \frac{1}{2}\langle U_i \rangle + \frac{\langle U_i^E \rangle + \langle U_i^W \rangle + \langle U_i^N \rangle + \langle U_i^S \rangle + \langle U_i^T \rangle + \langle U_i^B \rangle}{12}    
\end{equation}

This local filtering operation is highly compatible with the LBM's neighbor-only communication pattern, maintaining the algorithm's high parallel efficiency while providing a robust estimate of local turbulence intensity.

\subsubsection{Stochastic Integration and Stability}
The stochastic component of the LSM involves the integration of Gaussian white noise, $\xi(t)$, which is formally defined as the time derivative of a Wiener process $W(t)$:

\begin{equation}
    W_i(t) = \int_0^t \xi(s) ds    
\end{equation}

In accordance with Brownian motion theory, the variance of this process increases linearly with time. Given that the SGS velocity $u_s$ is modeled as a random walk, the stochastic term $(f_s C_0 \epsilon)^{1/2} d\xi$ can occasionally produce unrealistically high instantaneous velocities. This is particularly prevalent near building walls where $e_s$ is high and the gradient terms in the LSM equation are steep. To prevent numerical instability and maintain physical consistency, a velocity limitation is enforced, ensuring that the total Lagrangian particle velocity $u_p$ remains within physically plausible bounds relative to the local mean flow \cite{Yokouchi2020Simulation}.

\subsubsection{Model Selection and Validation via Flat Plate Analysis}

Before applying the model to complex urban morphologies, \textcite{Yokouchi2020Simulation} conducted a validation study using particle dispersion over a flat plate to identify the most accurate combination of numerical schemes. The evaluation compared three primary configurations: (1) a baseline model using a simple Bounce-Back (BB) boundary and the standard subgrid-scale (SGS) model; (2) a model incorporating the Wall Function (WF) and Suga's refined SGS TKE calculation; and (3) an integrated model combining the WF-Suga framework with the Lagrangian Stochastic Model (LSM) and a velocity limitation.The accuracy of these models was evaluated based on their ability to replicate the theoretical logarithmic profile of particle concentration in a neutral boundary layer:

\begin{equation}
    \frac{C_0 - C}{C^*} = \frac{1}{\kappa} \ln \frac{z}{z_0}    
\end{equation}

where $C_0$ represents the particle density at ground level, $C^*$ is the friction density, $\kappa$ is the von Kármán constant, and $z_0$ is the roughness length.The results demonstrated that the WF-Suga-LSM model provided the highest correlation with the logarithmic profile. In contrast, models without the LSM or the specialized SGS TKE tended to deviate from the expected distribution, particularly in the near-wall region. Furthermore, the necessity of the speed limitation—defined as $|u_s| < k_{sgs}^{1/2}$—was confirmed through an analysis of the particle velocity variance. Without this limitation, the stochastic term in the LSM produced unrealistically high velocity fluctuations near the boundary. Implementing the speed limit successfully reduced this variance, ensuring a stable and physically consistent velocity distribution across the vertical profile.By adopting this validated WF-Suga-LSM framework with speed limitations, this research ensures that the fundamental transport mechanisms are correctly modeled before introducing the complexities of the AIJ Case H urban block.

\newpage